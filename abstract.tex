\chapter*{Abstract}
\addcontentsline{toc}{chapter}{Abstract}

Helicopters are sometimes hit by lightning when flying offshore along the coast of Norway. Often, the helicopter's presence is what triggers the lightning strike, and this phenomenon is called Helicopter Triggered Lightning (HTL). These lightning strikes present both an economic and a safety risk to offshore operators. The current forecast for HTL in Norway was introduced in 2016, and named Helicopter Trigger Index. Since then, cases of HTL have been reported, which implies that the introduction of the forecast did not provide a robust enough forecast to prevent all HTL events. 

This thesis provides a thorough investigation into reported incidents of triggered lightning in Norway. In addition, the available theoretical models used as a basis for the HTL forecast are assessed.  The thesis reaffirms the importance of the 0$^{\circ}C$ isotherm in forecasting HTL. It is found that the Helicopter Trigger Index might be improved upon by increasing the weighting of current precipitation and temperature parameters when computing the index. It is also concluded that a possible predictor of HTL is wind directed on-shore, which is a parameter not included in the index today.

The study found also an error in the forecasting algorithm, leading to an overestimation of risk related to offshore flying. It was found that the correction of the algorithm error would increase the forecast skill, especially on the northern facing coast of Norway.