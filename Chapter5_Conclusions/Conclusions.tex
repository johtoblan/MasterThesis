\chapter{Conclusions}
\section{Are there still cases of Helicopter Triggered Lightning in Norway?}
\acrlong{htl} is still a recurring phenomenon, even after several years of forecasting the \acrlong{hti}. This is the reason why it is necessary to continue researching the phenomenon - the ideal scenario would be a forecast which prevented such cases altogether. 

This thesis found there to be an increase in cases in Norway after forecasting began in 2016. However, this may be seen as an artificial increase, as helicopter operators were in direct dialogue with the Meteorological Institute and more interested in reporting cases in order to improve the forecasting ability. It should also be noted that during the forecast period there have been no \textit{major} incidents. The cases happening were also found to be happening at lower temperatures than the expected 0$^{\circ}C$ isotherm. This was discussed in Section \ref{sec:avinor}: It is probably the result of forecasting removing the theoretical peak at 0$^{\circ}C$, where more major incidents would have happened.

\section{What meteorological phenomena are present during triggered lightning incidents?}
To identify atmospheric conditions leading to \acrshort{htl}, both \acrshort{metar} and composite plots were studied. This thesis found from both \acrshort{metar} and composite plots that convective precipitation and non-stratiform cloud types are related to triggered lightning incidents. This thesis also reaffirms that the temperature during triggered lightning events were situated around -3 to 0$^{\circ}C$ for the altitude of the aircraft, both for the fixed wing and for the helicopter situation. It also found typical pressure patterns leading to geostrophic winds incident on coastal areas, suggesting convection due to ocean-land roughness effects being an important factor in triggered lightning incidents. This ocean-land roughness effect may also contribute to Flesland having more triggered lightning incidents than Sola, due to Flesland being surrounded by more mountainous landscape.

\section{In what ways can the HTL forecast be improved?}
By implementing a decomposition strategy, this thesis found the precipitation and temperature sub-indices in the \acrlong{hti} to be quite good at forecasting \acrshort{htl}. The vertical velocity sub-index along with the cloud sub-index may have reduced the risk level from Red to Yellow in seven of the 11 cases studied. 

These results imply that improvements can be made to the \acrshort{hti} by weighting the precipitation and temperature sub-indices. Also, a reduction in the upper threshold of the vertical velocity criteria should be considered, as there is present a positive vertical velocity in all the recorded triggered lightning incidents. 

By reintroducing the hourly precipitation, there seems to be no severe risk increase, but rather a skill increase of the \acrshort{hti} forecast, such that an implementation of this fix would improve the forecast ability. This thesis was not able to perform an investigation into whether ensemble systems were able to increase the forecast skill, though preliminary efforts into this is expected to help in cases where precipitation was missing. There is also a known underestimation of coastal precipitation in the \acrshort{meps} system, such that investigations into increasing the precipitation threshold should be done when this known underestimation has been corrected. Lastly, including the 0$^{\circ}C$ isotherm in the operational forecast should be considered, but further sensitivity tests are necessary. As discussed in Section \ref{sec:plvsml}, the current interpolation scheme for the temperature is for a coarse vertical area based on the findings of \cite{wilkinson2013}. Investigations into improving this by using model level interpolation instead of the current pressure levels may give increased ability, as the difference in case temperatures between the finer \acrshort{era5} pressure levels and coarser \acrshort{meps} pressure levels were substantial.



