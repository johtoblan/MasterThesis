\chapter{Conclusions}
\section{Are there still cases of Helicopter triggered lightning?}
\acrlong{htl} is still a recurring phenomenon, even after several years of forecasting the \acrlong{hti}. This is the reason why it is necessary to continue researching the phenomenon - the ideal scenario would be a forecast which prevented such cases altogether. 

This thesis found there to be an increase in cases in Norway after forecasting began in 2016. However, this may be seen as an artificial increase, as helicopter operators were in direct dialogue with the Meteorological Institute and more interested in reporting cases in order to improve the forecasting ability. It should also be noted that during the forecast period there have been no major incidents leading to major incidents. The cases happening were also found to be happening at lower temperatures than the expected 0$^{\circ}C$ isotherm. This was discussed in Section \ref{sec:avinor}: It is probably the result of forecasting removing the theoretical peak at 0$^{\circ}C$, where more major incidents would have happened.

\section{What meteorological phenomena are observed during triggered lightning incidents?}
To identify atmospheric conditions leading to a \acrshort{htl}, both METAR and composite plots were studied. This thesis found from both METAR and composite plots that convective precipitation and non-stratiform cloud types to be related to triggered lightning incidents. This thesis also reaffirms that the temperature during triggered lightning events were situated around -3 to 0$^{\circ}C$ for the altitude of the aircraft, both for the fixed wing and for the helicopter situation. It also found typical pressure patterns leading to geostrophic winds incident on coastal areas, suggesting convection due to convergence being important.

\section{How does the Helicopter Trigger Index perform when cases are happening?}

\section{In what ways can the HTL-forecast be improved?}

\section{Future research}